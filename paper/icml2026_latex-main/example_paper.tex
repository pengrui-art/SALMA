%%%%%%%% ICML 2026 EXAMPLE LATEX SUBMISSION FILE %%%%%%%%%%%%%%%%%

\documentclass{article}

% Recommended, but optional, packages for figures and better typesetting:
\usepackage{microtype}
\usepackage{graphicx}
\usepackage{subcaption}
\usepackage{booktabs} % for professional tables

% hyperref makes hyperlinks in the resulting PDF.
% If your build breaks (sometimes temporarily if a hyperlink spans a page)
% please comment out the following usepackage line and replace
% \usepackage{icml2026} with \usepackage[nohyperref]{icml2026} above.
\usepackage{hyperref}


% Attempt to make hyperref and algorithmic work together better:
\newcommand{\theHalgorithm}{\arabic{algorithm}}

% Use the following line for the initial blind version submitted for review:
\usepackage{icml2026}

% For preprint, use
% \usepackage[preprint]{icml2026}

% If accepted, instead use the following line for the camera-ready submission:
% \usepackage[accepted]{icml2026}

\usepackage{amsmath}
\usepackage{amssymb}
\usepackage{mathtools}
\usepackage{amsthm}
\usepackage{subfigure}
\usepackage{longtable}
\usepackage{booktabs} 
\usepackage{multirow}

% if you use cleveref..
\usepackage[capitalize,noabbrev]{cleveref}

%%%%%%%%%%%%%%%%%%%%%%%%%%%%%%%%
% THEOREMS
%%%%%%%%%%%%%%%%%%%%%%%%%%%%%%%%
\theoremstyle{plain}
\newtheorem{theorem}{Theorem}[section]
\newtheorem{proposition}[theorem]{Proposition}
\newtheorem{lemma}[theorem]{Lemma}
\newtheorem{corollary}[theorem]{Corollary}
\theoremstyle{definition}
\newtheorem{definition}[theorem]{Definition}
\newtheorem{assumption}[theorem]{Assumption}
\theoremstyle{remark}
\newtheorem{remark}[theorem]{Remark}

% Todonotes is useful during development; simply uncomment the next line
%    and comment out the line below the next line to turn off comments
%\usepackage[disable,textsize=tiny]{todonotes}
\usepackage[textsize=tiny]{todonotes}

% The \icmltitle you define below is probably too long as a header.
% Therefore, a short form for the running title is supplied here:
\icmltitlerunning{Submission and Formatting Instructions for ICML 2026}

\begin{document}

\twocolumn[
  \icmltitle{Submission and Formatting Instructions for \\
    International Conference on Machine Learning (ICML 2026)}

  % It is OKAY to include author information, even for blind submissions: the
  % style file will automatically remove it for you unless you've provided
  % the [accepted] option to the icml2026 package.

  % List of affiliations: The first argument should be a (short) identifier you
  % will use later to specify author affiliations Academic affiliations
  % should list Department, University, City, Region, Country Industry
  % affiliations should list Company, City, Region, Country

  % You can specify symbols, otherwise they are numbered in order. Ideally, you
  % should not use this facility. Affiliations will be numbered in order of
  % appearance and this is the preferred way.
  \icmlsetsymbol{equal}{*}

  \begin{icmlauthorlist}
    \icmlauthor{Firstname1 Lastname1}{equal,yyy}
    \icmlauthor{Firstname2 Lastname2}{equal,yyy,comp}
    \icmlauthor{Firstname3 Lastname3}{comp}
    \icmlauthor{Firstname4 Lastname4}{sch}
    \icmlauthor{Firstname5 Lastname5}{yyy}
    \icmlauthor{Firstname6 Lastname6}{sch,yyy,comp}
    \icmlauthor{Firstname7 Lastname7}{comp}
    %\icmlauthor{}{sch}
    \icmlauthor{Firstname8 Lastname8}{sch}
    \icmlauthor{Firstname8 Lastname8}{yyy,comp}
    %\icmlauthor{}{sch}
    %\icmlauthor{}{sch}
  \end{icmlauthorlist}

  \icmlaffiliation{yyy}{Department of XXX, University of YYY, Location, Country}
  \icmlaffiliation{comp}{Company Name, Location, Country}
  \icmlaffiliation{sch}{School of ZZZ, Institute of WWW, Location, Country}

  \icmlcorrespondingauthor{Firstname1 Lastname1}{first1.last1@xxx.edu}
  \icmlcorrespondingauthor{Firstname2 Lastname2}{first2.last2@www.uk}

  % You may provide any keywords that you find helpful for describing your
  % paper; these are used to populate the "keywords" metadata in the PDF but
  % will not be shown in the document
  \icmlkeywords{Machine Learning, ICML}

  \vskip 0.3in
]

% this must go after the closing bracket ] following \twocolumn[ ...

% This command actually creates the footnote in the first column listing the
% affiliations and the copyright notice. The command takes one argument, which
% is text to display at the start of the footnote. The \icmlEqualContribution
% command is standard text for equal contribution. Remove it (just {}) if you
% do not need this facility.

% Use ONE of the following lines. DO NOT remove the command.
% If you have no special notice, KEEP empty braces:
\printAffiliationsAndNotice{}  % no special notice (required even if empty)
% Or, if applicable, use the standard equal contribution text:
% \printAffiliationsAndNotice{\icmlEqualContribution}

\begin{abstract}
  This document provides a basic paper template and submission guidelines.
  Abstracts must be a single paragraph, ideally between 4--6 sentences long.
  Gross violations will trigger corrections at the camera-ready phase.
\end{abstract}

\section{Introduction}

\section{Related Work}

\subsection{Paragraphs and Footnotes}

You can use footnotes\footnote{Footnotes should be complete sentences.}
to provide readers with additional information about a topic without
interrupting the flow of the paper. Indicate footnotes with a number in the
text where the point is most relevant. Place the footnote in 9~point type at
the bottom of the column in which it appears. Precede the first footnote in a
column with a horizontal rule of 0.8~inches.\footnote{Multiple footnotes can
  appear in each column, in the same order as they appear in the text,
  but spread them across columns and pages if possible.}

\begin{figure}[ht]
  \vskip 0.2in
  \begin{center}
    \centerline{\includegraphics[width=\columnwidth]{figs/icml_numpapers.pdf}}
    \caption{
      Historical locations and number of accepted papers for International
      Machine Learning Conferences (ICML 1993 -- ICML 2008) and International
      Workshops on Machine Learning (ML 1988 -- ML 1992). At the time this
      figure was produced, the number of accepted papers for ICML 2008 was
      unknown and instead estimated.
    }
    \label{icml-historical}
  \end{center}
\end{figure}

\subsection{Figures}

Number figures sequentially, placing the figure number and caption \emph{after}
the graphics, with at least 0.1~inches of space before the caption and
0.1~inches after it, as in \cref{icml-historical}. The figure caption should be
set in 9~point type and centered unless it runs two or more lines, in which
case it should be flush left. You may float figures to the top or bottom of a
column, and you may set wide figures across both columns (use the environment
\texttt{figure*} in \LaTeX). Always place two-column figures at the top or
bottom of the page.

\subsection{Algorithms}

If you are using \LaTeX, please use the ``algorithm'' and ``algorithmic''
environments to format pseudocode. These require the corresponding stylefiles,
algorithm.sty and algorithmic.sty, which are supplied with this package.
\cref{alg:example} shows an example.

\begin{algorithm}[tb]
  \caption{Bubble Sort}
  \label{alg:example}
  \begin{algorithmic}
    \STATE {\bfseries Input:} data $x_i$, size $m$
    \REPEAT
    \STATE Initialize $noChange = true$.
    \FOR{$i=1$ {\bfseries to} $m-1$}
    \IF{$x_i > x_{i+1}$}
    \STATE Swap $x_i$ and $x_{i+1}$
    \STATE $noChange = false$
    \ENDIF
    \ENDFOR
    \UNTIL{$noChange$ is $true$}
  \end{algorithmic}
\end{algorithm}


\subsection{Tables}

You may also want to include tables that summarize material. Like figures,
these should be centered, legible, and numbered consecutively. However, place
the title \emph{above} the table with at least 0.1~inches of space before the
title and the same after it, as in \cref{sample-table}. The table title should
be set in 9~point type and centered unless it runs two or more lines, in which
case it should be flush left.

% Note use of \abovespace and \belowspace to get reasonable spacing
% above and below tabular lines.

\begin{table}[t]
  \caption{Classification accuracies for naive Bayes and flexible
    Bayes on various data sets.}
  \label{sample-table}
  \begin{center}
    \begin{small}
      \begin{sc}
        \begin{tabular}{lcccr}
          \toprule
          Data set  & Naive         & Flexible      & Better?  \\
          \midrule
          Breast    & 95.9$\pm$ 0.2 & 96.7$\pm$ 0.2 & $\surd$  \\
          Cleveland & 83.3$\pm$ 0.6 & 80.0$\pm$ 0.6 & $\times$ \\
          Glass2    & 61.9$\pm$ 1.4 & 83.8$\pm$ 0.7 & $\surd$  \\
          Credit    & 74.8$\pm$ 0.5 & 78.3$\pm$ 0.6 &          \\
          Horse     & 73.3$\pm$ 0.9 & 69.7$\pm$ 1.0 & $\times$ \\
          Meta      & 67.1$\pm$ 0.6 & 76.5$\pm$ 0.5 & $\surd$  \\
          Pima      & 75.1$\pm$ 0.6 & 73.9$\pm$ 0.5 &          \\
          Vehicle   & 44.9$\pm$ 0.6 & 61.5$\pm$ 0.4 & $\surd$  \\
          \bottomrule
        \end{tabular}
      \end{sc}
    \end{small}
  \end{center}
  \vskip -0.1in
\end{table}

\subsection{Theorems and Such}
The preferred way is to number definitions, propositions, lemmas, etc.
consecutively, within sections, as shown below.
\begin{definition}
  \label{def:inj}
  A function $f:X \to Y$ is injective if for any $x,y\in X$ different, $f(x)\ne
    f(y)$.
\end{definition}
Using \cref{def:inj} we immediate get the following result:
\begin{proposition}
  If $f$ is injective mapping a set $X$ to another set $Y$,
  the cardinality of $Y$ is at least as large as that of $X$
\end{proposition}
\begin{proof}
  Left as an exercise to the reader.
\end{proof}
\cref{lem:usefullemma} stated next will prove to be useful.
\begin{lemma}
  \label{lem:usefullemma}
  For any $f:X \to Y$ and $g:Y\to Z$ injective functions, $f \circ g$ is
  injective.
\end{lemma}
\begin{theorem}
  \label{thm:bigtheorem}
  If $f:X\to Y$ is bijective, the cardinality of $X$ and $Y$ are the same.
\end{theorem}
An easy corollary of \cref{thm:bigtheorem} is the following:
\begin{corollary}
  If $f:X\to Y$ is bijective,
  the cardinality of $X$ is at least as large as that of $Y$.
\end{corollary}
\begin{assumption}
  The set $X$ is finite.
  \label{ass:xfinite}
\end{assumption}
\begin{remark}
  According to some, it is only the finite case (cf. \cref{ass:xfinite}) that
  is interesting.
\end{remark}
%restatable

\subsection{Citations and References}


Citations within the text should include the authors' last names and year. If
the authors' names are included in the sentence, place only the year in
parentheses, for example when referencing Arthur Samuel's pioneering work
\yrcite{Samuel59}. Otherwise place the entire reference in parentheses with the
authors and year separated by a comma \cite{Samuel59}. List multiple references
separated by semicolons \cite{kearns89,Samuel59,mitchell80}. Use the `et~al.'
construct only for citations with three or more authors or after listing all
authors to a publication in an earlier reference \cite{MachineLearningI}.

Authors should cite their own work in the third person in the initial version
of their paper submitted for blind review. Please refer to \cref{author info}
for detailed instructions on how to cite your own papers.

Use an unnumbered first-level section heading for the references, and use a
hanging indent style, with the first line of the reference flush against the
left margin and subsequent lines indented by 10 points. The references at the
end of this document give examples for journal articles \cite{Samuel59},
conference publications \cite{langley00}, book chapters \cite{Newell81}, books
\cite{DudaHart2nd}, edited volumes \cite{MachineLearningI}, technical reports
\cite{mitchell80}, and dissertations \cite{kearns89}.

\section{Preliminaries}

\section{Methodology}

\section{Experiment}


\section{Conclusion}


\section*{Accessibility}

Authors are kindly asked to make their submissions as accessible as possible
for everyone including people with disabilities and sensory or neurological
differences. Tips of how to achieve this and what to pay attention to will be
provided on the conference website \url{http://icml.cc/}.

\section*{Software and Data}

If a paper is accepted, we strongly encourage the publication of software and
data with the camera-ready version of the paper whenever appropriate. This can
be done by including a URL in the camera-ready copy. However, \textbf{do not}
include URLs that reveal your institution or identity in your submission for
review. Instead, provide an anonymous URL or upload the material as
``Supplementary Material'' into the OpenReview reviewing system. Note that
reviewers are not required to look at this material when writing their review.

% Acknowledgements should only appear in the accepted version.
\section*{Acknowledgements}

\textbf{Do not} include acknowledgements in the initial version of the paper
submitted for blind review.

If a paper is accepted, the final camera-ready version can (and usually should)
include acknowledgements.  Such acknowledgements should be placed at the end of
the section, in an unnumbered section that does not count towards the paper
page limit. Typically, this will include thanks to reviewers who gave useful
comments, to colleagues who contributed to the ideas, and to funding agencies
and corporate sponsors that provided financial support.

\section*{Impact Statement}

Authors are \textbf{required} to include a statement of the potential broader
impact of their work, including its ethical aspects and future societal
consequences. This statement should be in an unnumbered section at the end of
the paper (co-located with Acknowledgements -- the two may appear in either
order, but both must be before References), and does not count toward the paper
page limit. In many cases, where the ethical impacts and expected societal
implications are those that are well established when advancing the field of
Machine Learning, substantial discussion is not required, and a simple
statement such as the following will suffice:

``This paper presents work whose goal is to advance the field of Machine
Learning. There are many potential societal consequences of our work, none
which we feel must be specifically highlighted here.''

The above statement can be used verbatim in such cases, but we encourage
authors to think about whether there is content which does warrant further
discussion, as this statement will be apparent if the paper is later flagged
for ethics review.

% In the unusual situation where you want a paper to appear in the
% references without citing it in the main text, use \nocite
\nocite{langley00}

\bibliography{example_paper}
\bibliographystyle{icml2026}

%%%%%%%%%%%%%%%%%%%%%%%%%%%%%%%%%%%%%%%%%%%%%%%%%%%%%%%%%%%%%%%%%%%%%%%%%%%%%%%
%%%%%%%%%%%%%%%%%%%%%%%%%%%%%%%%%%%%%%%%%%%%%%%%%%%%%%%%%%%%%%%%%%%%%%%%%%%%%%%
% APPENDIX
%%%%%%%%%%%%%%%%%%%%%%%%%%%%%%%%%%%%%%%%%%%%%%%%%%%%%%%%%%%%%%%%%%%%%%%%%%%%%%%
%%%%%%%%%%%%%%%%%%%%%%%%%%%%%%%%%%%%%%%%%%%%%%%%%%%%%%%%%%%%%%%%%%%%%%%%%%%%%%%
\newpage
\appendix
\onecolumn
\section{You \emph{can} have an appendix here.}

You can have as much text here as you want. The main body must be at most $8$
pages long. For the final version, one more page can be added. If you want, you
can use an appendix like this one.

The $\mathtt{\backslash onecolumn}$ command above can be kept in place if you
prefer a one-column appendix, or can be removed if you prefer a two-column
appendix.  Apart from this possible change, the style (font size, spacing,
margins, page numbering, etc.) should be kept the same as the main body.
%%%%%%%%%%%%%%%%%%%%%%%%%%%%%%%%%%%%%%%%%%%%%%%%%%%%%%%%%%%%%%%%%%%%%%%%%%%%%%%
%%%%%%%%%%%%%%%%%%%%%%%%%%%%%%%%%%%%%%%%%%%%%%%%%%%%%%%%%%%%%%%%%%%%%%%%%%%%%%%

\end{document}

% This document was modified from the file originally made available by
% Pat Langley and Andrea Danyluk for ICML-2K. This version was created
% by Iain Murray in 2018, and modified by Alexandre Bouchard in
% 2019 and 2021 and by Csaba Szepesvari, Gang Niu and Sivan Sabato in 2022.
% Modified again in 2023 and 2024 by Sivan Sabato and Jonathan Scarlett.
% Previous contributors include Dan Roy, Lise Getoor and Tobias
% Scheffer, which was slightly modified from the 2010 version by
% Thorsten Joachims & Johannes Fuernkranz, slightly modified from the
% 2009 version by Kiri Wagstaff and Sam Roweis's 2008 version, which is
% slightly modified from Prasad Tadepalli's 2007 version which is a
% lightly changed version of the previous year's version by Andrew
% Moore, which was in turn edited from those of Kristian Kersting and
% Codrina Lauth. Alex Smola contributed to the algorithmic style files.
